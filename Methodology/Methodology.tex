\section{Methodology}
\label{sec:methodology}
	To determine the feasibility of this approach to intelligent switching the team will simulate the behavior of a network of intelligent switches in software.
	The team will construct a simulated network consisting of virtual machines capable of creating network traffic.
	The virtual machines will perform a series of predetermined requests through a virtual router to servers both inside and outside of the network.
	We will also have servers on the virtual network that will receive requests from other hosts on the network as well as foreign hosts outside the network.
	We will verify the functionality of our system by confirming that the requests and responses are correct.
	The effectiveness of the system will be measured by gaging the latency and throughput of the network in forwarding the requests.
	
	Using Python and the impacket library we will create a switch emulator using a virtual machine with enough network interfaces to support the topologies we will test.
	We will also simulate the behavior of a conventional switch to demonstrate that our approach works in conjunction with these switches, as well as to demonstrate the advantage of our switch over a conventional device.
	
	Several topologies will be employed in testing our system.
	We will start with a single smart switch in a star topology (to test its basic functionality as a switch), then move to one of our switches along with a conventional switch in a loop topology.
	We can also test with two conventional switches in the same configuration, to demonstrate that a broadcast storm would occur under these conditions.
	Next, we will move to a topology involving multiple smart switches connected to each other.
	Our final test will simulate a loop topology that exceeds the capacity of the circular buffer to ensure that the switch can still fall back to disabling loop links even if it cannot keep track of the entire network topology.
