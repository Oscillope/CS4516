\documentclass{article}
\usepackage[T1]{fontenc}
\usepackage[utf8]{inputenc}
\usepackage[section]{placeins}
\usepackage{fullpage}

\title{CS 4516 Project Proposal\\Group \#5: Bandwidth Trunking Using Layer 2 Devices}
\author{Jason Rosenman \and Louis Fogel \and Sam Abradi}
\date{}

\begin{document}
\maketitle
\begin{abstract}
	Our project aims to address a need for more resilient, higher performance Layer 2 with only one piece of additional hardware by allowing multiple switch ports to be connected without the risk of forwarding loops or broadcast storms.
	In addition, we aim to allow switches to distribute bandwidth over multiple interfaces when connected together.
	This allows us to not only be competitive with systems such as Etherfuse \cite{etherfuse}, which prevents forwarding loops by disabling ports that it detects are linked, but to surpass them by adding enhanced redundancy and performance via bandwidth sharing.
\end{abstract}
\section{Need}
	Except for a in few special applications, Ethernet is far and away the most popular choice of link layer protocol.
	In consumer and enterprise level networks there are no real viable alternatives.
	Ethernet’s popularity stems from its simplicity and versatility, but it is not without shortcomings.

	Typical modern installations of switched Ethernet are highly susceptible to broadcast storms created by forwarding loops.
	This problem is especially important in networks with publicly accessible infrastructure such as schools and hotels.
	Intelligent switching hardware could prevent these broadcast storms from occurring, and additionally, could provide better reliability and reduce bottlenecks by allowing redundant links and bandwidth sharing between routes.
	
	In commercial systems, reliability and fool-proofing are important.
	With Ethernet as it currently stands (without intelligent switching hardware), a malicious or simply confused user could easily create broadcast storms or forwarding loops simply by connecting a switch to itself or directly to another switch that is already connected to that switch.
	These storms could easily bring down all or part of a large network.
	Therefore, there is a need for a system to address and resolve this security and performance risk.
	
	Solutions to these problems are either too slow to allow full featured link redundancy or prohibitively expensive and involve entirely replacing network infrastructure.
	Additionally, some of these devices depend on external configuration by administrators, incurring additional setup  and maintenance costs.
	Our approach allows link sharing and broadcast storm mitigation in switched Ethernet networks using a single self-configuring device.
\section{Approach}
	Our team proposes the creation an Ethernet switch that implements flow-based forwarding.
	Unlike a conventional switch, which uses a forwarding table to determine which interface to forward frames to, our system would create flows based on source IP address (when available) to allow bandwidth sharing over linked interfaces.
	Even when linked with a conventional switch, our device could utilize multiple links to share bandwidth upstream (towards the linked switch) and have redundancy downstream (from the linked switch).
	Assuming that the conventional switch would forward to multiple ports simultaneously if it detects the same MAC address multiple times, a link could fail between our switch and the conventional switch the loss of a single frame.
	As a result, ongoing connections can be maintained throughout a link failure.

	We propose detection forwarding loops by caching a hash of each frame that is received, and comparing it to incoming frames to detect duplicates.
	To increase efficiency and limit hardware complexity, we propose using the CRC that is already included in the Ethernet frame as the hash.
	We propose calculation the necessary size of the cache using estimated time-to-live (TTL) values for the CRCs.
	The TTL would be a function of the maximum number of hops in a loop, and of the maximum amount of broadcast traffic.
	These maximums would be practical limitations on the system we are interested in supporting, as opposed to theoretical limitations.
	In the case that these maximums are exceeded, our device could fall back to simply disabling redundant links when a loop is detected, thereby turning the network into a directed acyclic graph.
	The looped interface will remain disabled until the state of the network falls within the limits of the cache.
	
	When no redundant links exist, the created flows would essentially form a forwarding table.
	However, when redundant links are formed, the device could use the source IP address of the packet to make flows that determine forwarding strategy for frames.
	We propose creating flows based on measured bandwidth of interfaces with the goal of uniform utilization.
	Our device could detect link failures when a previously existing loop no longer exists.
	In this condition, the device would delete all flows related to that loop.
	This would cause it to rebuild new flows using the remaining available links.

	To determine the feasibility of this approach to intelligent switching the team proposes simulation of the behavior of a network of intelligent switches in software.
	The team would construct a simulated network consisting of virtual machines capable of creating network traffic.
	The virtual machines will perform a series of predetermined requests through a virtual router to servers both inside and outside of the network.
	We will also have servers on the virtual network that will receive requests from other hosts on the network as well as foreign hosts outside the network.
	Several topologies and scenarios would be employed in testing our system to show a range of behaviors in the device.
	We can also test with two conventional "dumb" switches in the same configuration, to demonstrate that a broadcast storm would occur under these conditions.
	We propose verification the functionality of our system by confirming that the requests and responses are correct.
	The effectiveness of the system would be measured by gaging the latency and throughput of the network in forwarding the requests.
\end{document}
\section{Benefit}
	Using this information, intelligent switches will be able to detect and avoid link failures and share bandwidth between redundant links to mitigate bottlenecks.
	Because switches on this protocol will still be entirely compliant with existing Ethernet networks, this approach will also allow switches that are not able to detect loops to operate in the same network.
	If non-intelligent switches are used in combination with at least one intelligent switch and the network is small enough to allow CRC caching, our hardware should still be able to share bandwidth by essentially tricking the non-intelligent switch into thinking that it is two switches attached to separate ports.
	Since our switches are auto-configuring as well as backwards compatible with existing hardware, implementing them in a network requires a very low installation and maintenance overhead.
\section{Competition}
	Previous research has identified methods of detecting and eliminating routing loops in switched Ethernet networks but falls short of network topology detection for link redundancy or bandwidth sharing.
	One such research paper is Etherfuse \cite{etherfuse}, which documents a system by which a switch would be able to detect forwarding loops and graph structure in an Ethernet network.
	The paper only suggests that the switch could then disable the link that was creating the forwarding loop, as opposed to using it to increase bandwidth between switches.
	
\begin{thebibliography}{9}
\bibitem{etherfuse}
	Khaled Elmeleegy, Alan L. Cox, T. S. Eugene Ng,
	\emph{EtherFuse: An Ethernet Watchdog}.
	Rice University, 2007.
\end{thebibliography}
\end{document}
