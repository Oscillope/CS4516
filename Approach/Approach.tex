\section{Approach}
\label{sec:approach}
    Our team proposes the creation an Ethernet switch that implements flow-based forwarding.
    Unlike a conventional switch, which uses a forwarding table to determine which interface to forward frames to, our system would create flows based on source IP address (when available) to allow bandwidth sharing over linked interfaces.
    Even when linked with a conventional switch, our device could utilize multiple links to share bandwidth upstream (towards the linked switch) and have redundancy downstream (from the linked switch).
    Assuming that the conventional switch would forward to multiple ports simultaneously if it detects the same MAC address multiple times, a link could fail between our switch and the conventional switch the loss of a single frame.
    As a result, ongoing connections can be maintained throughout a link failure.

    We propose detecting forwarding loops by caching a hash of each frame that is received, and comparing it to incoming frames to detect duplicates.
    To increase efficiency and limit hardware complexity, we can use the CRC that is already included in the Ethernet frame as the hash.
    We propose calculating the necessary size of the cache using estimated time-to-live (TTL) values for the CRCs.
    The TTL would be a function of the maximum number of hops in a loop, and of the maximum amount of broadcast traffic.
    These maximums would be practical limitations on the system we are interested in supporting, as opposed to theoretical limitations.
    In the case that these maximums are exceeded, our device could fall back to simply disabling redundant links when a loop is detected, thereby turning the network into a directed acyclic graph.
    The looped interface will remain disabled until the state of the network falls within the limits of the cache.

    When no redundant links exist, the created flows would essentially form a forwarding table.
    However, when redundant links are formed, the device could use the source IP address of the packet to make flows that determine forwarding strategy for frames.
    We propose creating flows based on measured bandwidth of interfaces with the goal of uniform utilization.
    Our device could detect link failures when a previously existing loop no longer exists.
    In this condition, the device would delete all flows related to that loop.
    This would cause it to rebuild new flows using the remaining available links.

    To determine the feasibility of this approach to intelligent switching the team proposes simulation of the behavior of a network of intelligent switches in software.
    The team would construct a simulated network consisting of virtual machines capable of creating network traffic.
    The virtual machines will perform a series of predetermined requests through a virtual router to servers both inside and outside of the network.
    We will also have servers on the virtual network that will receive requests from other hosts on the network as well as foreign hosts outside the network.
    Several topologies and scenarios would be employed in testing our system to show a range of behaviors in the device.
    We can also test with two conventional ``dumb'' switches in the same configuration, to demonstrate that a broadcast storm would occur under these conditions.
    We propose verification of the functionality of our system by confirming that the requests and responses are correct.
    The effectiveness of the system would be measured by gauging the latency and throughput of the network in forwarding the requests.
