\section{Approach}
\label{sec:approach}
    Unlike a conventional switch, which uses a forwarding table to determine which interface to forward frames, LoopAware creates flow-like entries in its forwarding table based on source IP address (when available) to allow bandwidth sharing over linked interfaces.
    Even when linked with a conventional switch, LoopAware can utilize multiple links to share bandwidth upstream (towards the linked switch) and have redundancy downstream (from the linked switch).
    Assuming that a conventional switch will forward to multiple ports simultaneously if it detects the same MAC address multiple times, a link can go down between LoopAware and the conventional switch without dropping a single frame.
    As a result, ongoing connections can be maintained throughout a link failure.

    A notable challenge to our implementation is the instantaneous detection of loops in the network.
    LoopAware detects these loops by caching a hash of each frame that is received, and comparing it to incoming frames to detect duplicates.
    When a duplicate frame is detected, the interface it was sent on and the interface it was received on are determined to be aliases of each other.
    In the case that a duplicate frame was a broadcast and the source interface can not be determined, LoopAware sends a frame back over the interface that the frame was received on.
    To increase efficiency and limit complexity in a hardware implementation of LoopAware, we would recommend using the CRC that is already included in the Ethernet frame as the hash.
    Due to the difficulty of obtaining the CRC from the frames in software, our implementation simply stores the entire packet for comparison.
    As new CRCs are added to the buffer, the older ones will be overwritten.
    The minimum size of the buffer is dependent on time-to-live (TTL) values for the CRCs.
    The TTL is a function of the maximum number of hops in a loop, and of the maximum amount of broadcast traffic.
    These maximums are practical limitations on the system we are interested in supporting, as opposed to theoretical limitations.
    In the case that these maximums are exceeded, LoopAware falls back to simply disabling redundant links when a loop is detected, thereby turning the network into a directed acyclic graph.
    The looped interface will remain disabled until the state of the network falls within the limits of the cache.
    
    When no redundant links exist, the created flows form a forwarding table like that of any other Ethernet switch.
    However, when redundant links are formed, the device will use the source IP address of the packet to make flows that determine forwarding strategy for frames.
    Flows are created based on measured bandwidth of interfaces with the goal of keeping utilization balanced.
    LoopAware detects link failures when a previously existing loop no longer exists.
    In this condition, the device deletes all flows related to that loop.
    This causes it to rebuild new flows using the remaining available links.
