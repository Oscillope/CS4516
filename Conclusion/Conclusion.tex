\section{Conclusion}
% Conclusion: Students should write a conclusion for the work, summarizing the contributions, the impact, and potential for follow-on work. This section will likely be at least half a page. 
  \subsection{Findings}
    Our intelligent switch implementation was able to successfully detect forwarding loops and prevent broadcast storms from occurring.
    We successfully met our goal of implementing a switch that is auto-configuring; the device does not need any configuration beyond being connected to the network.
    Our device was able to respond quickly to link failures, preventing significant packet loss.
    In addition, when loops were detected, the switch was able to create flows that would allow bandwidth to be shared over multiple links.
    This confirms that the behavior we describe creates an auto-configuring Ethernet switch capable of using its redundant links instead of simply disabling them.
    We believe that no such device has been presented before.
    
    Allowing bandwidth sharing prevents bottlenecks from forming in the forwarding path by balancing the load distributed between paths.
    This behavior is a significant improvement over STP, which does not share bandwidth, because STP may find a spanning tree that causes the majority of traffic to flow through a single switch.
    
    Detecting link failures and updating the forwarding table accordingly allows a significant performance advantage over approaches like STP, which would require a new spanning tree to be detected.
    Fast automatic response to link failure is a highly desirable attribute for a Layer 2 networking device because it prevents any significant downtime for the network, and does not require immediate action by network administrators.
    With our device when a link can be brought back up the loop will be detected immediately.
    No configuration will be necessary to return the network back to its original state.
  \subsection{Future Work}
    The main limitation of our study was the significantly decreased performance of simulating our switch entirely in software.
    Future work could explore implementing the behaviors we describe on an OpenFlow switching device.
    This would allow much more realistic simulations of performance by reducing the number of artificial bottlenecks.
    Ideally, a full hardware implementation of the device would be created on a rapid prototyping platform such as the NetFPGA device.
    Hardware implementations would allow for much greater parallelism, potentially even allowing multiple broadcast frames to be handled simultaneously.
    A hardware implementation of our device would use hash tables to match repeated broadcast frame CRCs in constant time.
    We believe that if our device were effectively implemented in hardware, the latency issues would become negligible and the performance would be more comparable to regular "dumb" Ethernet switches.

    In this paper we discuss the behavior of only one intelligent switch in combination with multiple "dumb" switches.
    Future work could explore the benefits of using multiple intelligent switches to allow coordination.
    Coordination between intelligent switches would allow for more complex bandwidth sharing techniques that would allow for minimizing latency in addition to avoiding congestion.
    Coordination would also allow preventing packet loss in duplex when a link failure is detected.
