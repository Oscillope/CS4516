\documentclass{article}
\usepackage[T1]{fontenc}
\usepackage[utf8]{inputenc}
\usepackage[section]{placeins}
\usepackage{fullpage}
\usepackage{hyperref}
\usepackage{graphicx}
\usepackage{multicol}
\usepackage{multirow}
\usepackage{caption}
\usepackage{subcaption}
\usepackage{pgfplots}
\pgfplotsset{compat=1.8}

\graphicspath{ {./images/} }
\hypersetup{colorlinks=true,
			urlcolor=blue,
			linktoc=all,
			linkcolor=black,
			citecolor=black}

\title{LoopAware: Bandwidth Trunking Using Layer 2 Devices}
\author{Jason Rosenman \and Louis Fogel \and Sam Abradi}
\date{}

\begin{document}
\maketitle
\begin{abstract}
	Our project aims to address a need for more resilient, higher performance layer 2 with only one piece of additional hardware by allowing multiple switch ports to be connected without the risk of forwarding loops or broadcast storms.
	In addition, we aim to allow switches to distribute bandwidth over multiple interfaces when connected together.
	This allows us to not only be competitive with systems such as EtherFuse \cite{etherfuse}, which prevents forwarding loops by disabling ports that it detects are linked, but to surpass them by adding enhanced redundancy and performance via bandwidth sharing.
\end{abstract}
\section{Introduction}
\label{sec:introduction}
	%Need
	This paper introduces LoopAware, a device that can be used to allow more complex topologies in Ethernet networks.
	Namely, it allows Ethernet networks to contain redundant links without creating broadcast storms or forwarding loops.
	Unlike other systems that can allow loops in Ethernet networks, LoopAware can use loops in networks for redundancy as well as bandwidth sharing.
	It is also auto-configuring, so it requires no user intervention in order to work.
	LoopAware is backwards compatible with non-LoopAware switches.
	
	Except for a in few special applications, Ethernet is far and away the most popular choice of link layer protocol.
	In consumer and enterprise level networks there are no real viable alternatives.
	Ethernet's popularity stems from its simplicity and versatility, but it is not without shortcomings.

	Typical modern installations of switched Ethernet are highly susceptible to broadcast storms created by forwarding loops.
	This problem is especially important in networks with publicly accessible infrastructure such as schools and hotels.
	Intelligent switching hardware could prevent these broadcast storms from occurring, and additionally, could provide better reliability and reduce bottlenecks by allowing redundant links and bandwidth sharing between routes.

	In commercial systems, reliability and fool-proofing are important.
	With Ethernet as it currently stands (without intelligent switching hardware), a malicious or simply confused user could easily create broadcast storms or forwarding loops simply by connecting a switch to itself or directly to another switch that is already connected to that switch.
	These storms could easily bring down all or part of a large network.
	Therefore, there is a need for a system to address and resolve this security and performance risk.
	
	%Competition
	Previous research has identified methods of detecting and eliminating routing loops in switched Ethernet networks but fails to automatically detect network topology for bandwidth sharing.
	Many proposed solutions require replacement of all network infrastructure, which is prohibitively expensive.
	Additionally, some require manual configuration of the network, costing time and allowing for misconfiguration that could cause network outages.

	One proposed solution to the problem of forwarding loops in layer 2 networks is Spanning Tree Protocol (STP).
	STP works by sending special frames, known as Bridge Protocol Data Units (BPDUs), to identify the topology of the network and identify a spanning tree and disables all connections except those in the spanning tree.
	STP has several shortcomings that make it an undesirable alternative to our solution.
	Mainly, it requires replacing the entire network infrastructure with STP-capable hardware, which is expensive and necessitates significant network downtime.
	In addition, STP works by disabling redundant connections, and as a result can not support bandwidth sharing.
	Because our system does not disable links (except when overloaded), it is able to send traffic to the same host over different connections, preventing bottlenecks from forming in the forwarding graph.
	In some network topologies, the use of STP can result in count-to-infinity conditions between hosts, causing some frames to become unforwardable.
	Our device is not dependent on a hop count and is therefore immune to the count-to-infinity problem.
	Because spanning tree detection in STP is dependent on the ability of switches to detect BPDUs, it does not completely prevent the possibility of forwarding loops.

	Our device builds in part on the design of a similar system, EtherFuse, which detects forwarding loops in an Ethernet network using STP.
	EtherFuse uses its capability to detect forwarding loops to disable redundant links in cases where STP fails to \cite{etherfuse}.
	Our device expands on the forwarding loop detection technique that EtherFuse proved effective, but provides additional functionality that is useful for administrators of enterprise networks.
	EtherFuse is only compatible with networks that already support Spanning Tree Protocol (STP).
	In addition to detecting loops, EtherFuse also keeps track of count-to-infinity conditions that can occur under STP.
	In the condition of a loop or count-to-infinity, the device would simply sever the connection on which it is placed.
	LoopAware makes several improvements to the functionality of EtherFuse.
	First, an EtherFuse device must be purchased for every link that may take part in a loop, while LoopAware allows the prevention of forwarding loops without the need for additional devices in certain topologies.
	Second, the EtherFuse will only sever the connection, as opposed to building on it as LoopAware does.

	OpenFlow is another technology that can be used to prevent forwarding loops in Ethernet networks.
	OpenFlow capable switching hardware allows for custom rules for the handling of flows instead of the traditional implementation of Ethernet packet forwarding \cite{openflow}.
	While OpenFlow allows for an impressive number of features and behaviors, and indeed, could even be used as a testbed for our behaviors, it is better suited to research networks and data centers than enterprise networks.
	The enhanced capability of OpenFlow comes at the high cost of replacing all network hardware with OpenFlow-capable devices.
	The extended network downtime necessitated by such an extreme infrastructure upgrade would also probably prove unacceptable for enterprise networks.
	Additionally, providing an OpenFlow device with the same rules that our proposed device would operate on requires manual configuration of the device (something our device would not require).
	Finally, because flow creation rules for OpenFlow are handled in software, an OpenFlow device that automatically prevents new loops from causing broadcast storms would be much slower to handle broadcast frames.

	LoopAware aims to be ideal for enterprise networks by combining the best of both worlds: the robust auto-configuration and cost-effectiveness of EtherFuse and the powerful traffic management of OpenFlow.
	Enterprise networks need to be cheap, reliable, and easy to maintain.
	LoopAware solves the problems and limitations of switched Ethernet without compromising the cost-effectiveness of commodity hardware, the reliability of Ethernet, or the auto-configuration of traditional frame forwarding.
	LoopAware allows for an easy migration to a more advanced layer 2 infrastructure without requiring a complete system overhaul.
	
	We believe LoopAware would have appeal to administrators of small to medium sized enterprise networks.
	When limited IT staff are available, the auto-configuration and robustness of our device are desirable.
	Additionally, the backwards compatibility should keep cost of integration with existing networks low.
	We believe there is a market for robust networks without the headaches of more complex solutions that are better suited for large networks and datacenters.
	
	In this paper we describe an implementation of a LoopAware device.
	The implementation is described in Section \ref{sec:approach}.
	We then test our implementation to demonstrate its efficacy and performance.
	The testing methodology is described in Section \ref{sec:methodology} and results are presented and analyzed in Section \ref{sec:results}.
	We conclude in Section \ref{sec:conclusion}.

\section{Approach}
\label{sec:approach}
    Unlike a conventional switch, which uses a forwarding table to determine which interface to forward frames, LoopAware creates flow-like entries in its forwarding table based on source IP address (when available) to allow bandwidth sharing over linked interfaces.
    Even when linked with a conventional switch, LoopAware can utilize multiple links to share bandwidth upstream (towards the linked switch) and have redundancy downstream (from the linked switch).
    Assuming that a conventional switch will forward to multiple ports simultaneously if it detects the same MAC address multiple times, a link can go down between LoopAware and the conventional switch without dropping a single frame.
    As a result, ongoing connections can be maintained throughout a link failure.

    A notable challenge to our implementation is the instantaneous detection of loops in the network.
    LoopAware detects these loops by caching a hash of each frame that is received, and comparing it to incoming frames to detect duplicates.
    When a duplicate frame is detected, the interface it was sent on and the interface it was received on are determined to be aliases of each other.
    In the case that a duplicate frame was a broadcast and the source interface can not be determined, LoopAware sends a frame back over the interface that the frame was received on.
    To increase efficiency and limit complexity in a hardware implementation of LoopAware, we would recommend using the CRC that is already included in the Ethernet frame as the hash.
    Due to the difficulty of obtaining the CRC from the frames in software, our implementation simply stores the entire packet for comparison.
    As new hashes are added to the buffer, the older ones will be removed after enough other packets have been added to the buffer.
    The minimum size of the buffer is dependent on time-to-live (TTL) values for the hashes.
    The TTL is a function of the maximum number of hops in a loop, and of the maximum amount of broadcast traffic.
    These maximums are practical limitations on the system we are interested in supporting, as opposed to theoretical limitations.
    In the case that these maximums are exceeded, LoopAware falls back to simply disabling redundant links when a loop is detected, thereby turning the network into a directed acyclic graph.
    The looped interface will remain disabled until the state of the network falls within the limits of the cache.
    
    When no redundant links exist, the created flows form a forwarding table like that of any other Ethernet switch.
    However, when redundant links are formed, the device will use the source IP address of the packet to make flows that determine forwarding strategy for frames.
    Flows are created based on measured bandwidth of interfaces with the goal of keeping utilization balanced.
    LoopAware detects link failures when a previously existing loop no longer exists.
    In this condition, the device deletes all flows related to that loop.
    This causes it to rebuild new flows using the remaining available links.

\section{Methodology}
\label{sec:methodology}
    We conducted a series of tests to determine the feasibility of LoopAware.
    All tests were conducted on the simulation of a LoopAware device we implemented in software and the corresponding simulation of a regular "dumb" Ethernet switch.
    We constructed a simulated network consisting of virtual machines capable of creating different types of network traffic.
    The virtual machines performed a series of predetermined requests through a virtual router to servers both inside and outside of the network.
    We determined the effectiveness of the system by gaging the latency and throughput of the network in forwarding the requests.

    Three topologies were be employed in testing LoopAware.
    For each topology, the network was tested with our simulated Ethernet switch and then with our simulated LoopAware device.
    We began with a single switch in a star topology to set a baseline for functionality as a switch, as shown in Figure \ref{fig:startop}.
    Next, we tested the switches in a configuration with redundant links to a conventional switch, creating a forwarding loop.
    This configuration is shown in Figure \ref{fig:looptop}.
    \begin{figure}[ht]
        \centering
        \begin{subfigure}[b]{0.4\textwidth}
            \centering
            % Graphic for TeX using PGF
% Title: /home/jason/Dropbox/CS/CS4516/Term_Project/images/topology2.dia
% Creator: Dia v0.97.2
% CreationDate: Sun Mar  2 22:49:27 2014
% For: jason
% \usepackage{tikz}
% The following commands are not supported in PSTricks at present
% We define them conditionally, so when they are implemented,
% this pgf file will use them.
\ifx\du\undefined
  \newlength{\du}
\fi
\setlength{\du}{15\unitlength}
\begin{tikzpicture}
\pgftransformxscale{1.000000}
\pgftransformyscale{-1.000000}
\definecolor{dialinecolor}{rgb}{0.000000, 0.000000, 0.000000}
\pgfsetstrokecolor{dialinecolor}
\definecolor{dialinecolor}{rgb}{1.000000, 1.000000, 1.000000}
\pgfsetfillcolor{dialinecolor}
\definecolor{dialinecolor}{rgb}{1.000000, 1.000000, 1.000000}
\pgfsetfillcolor{dialinecolor}
\fill (9.000000\du,9.000000\du)--(9.000000\du,11.000000\du)--(14.000000\du,11.000000\du)--(14.000000\du,9.000000\du)--cycle;
\pgfsetlinewidth{0.100000\du}
\pgfsetdash{}{0pt}
\pgfsetdash{}{0pt}
\pgfsetmiterjoin
\definecolor{dialinecolor}{rgb}{0.000000, 0.000000, 0.000000}
\pgfsetstrokecolor{dialinecolor}
\draw (9.000000\du,9.000000\du)--(9.000000\du,11.000000\du)--(14.000000\du,11.000000\du)--(14.000000\du,9.000000\du)--cycle;
% setfont left to latex
\definecolor{dialinecolor}{rgb}{0.000000, 0.000000, 0.000000}
\pgfsetstrokecolor{dialinecolor}
\node at (11.500000\du,10.195000\du){CS4516-04};
\definecolor{dialinecolor}{rgb}{1.000000, 1.000000, 1.000000}
\pgfsetfillcolor{dialinecolor}
\pgfpathellipse{\pgfpoint{18.523888\du}{7.500000\du}}{\pgfpoint{2.523888\du}{0\du}}{\pgfpoint{0\du}{1.500000\du}}
\pgfusepath{fill}
\pgfsetlinewidth{0.100000\du}
\pgfsetdash{}{0pt}
\pgfsetdash{}{0pt}
\pgfsetmiterjoin
\definecolor{dialinecolor}{rgb}{0.000000, 0.000000, 0.000000}
\pgfsetstrokecolor{dialinecolor}
\pgfpathellipse{\pgfpoint{18.523888\du}{7.500000\du}}{\pgfpoint{2.523888\du}{0\du}}{\pgfpoint{0\du}{1.500000\du}}
\pgfusepath{stroke}
% setfont left to latex
\definecolor{dialinecolor}{rgb}{0.000000, 0.000000, 0.000000}
\pgfsetstrokecolor{dialinecolor}
\node at (18.523888\du,7.695000\du){CS4516-01};
\definecolor{dialinecolor}{rgb}{1.000000, 1.000000, 1.000000}
\pgfsetfillcolor{dialinecolor}
\fill (19.000000\du,13.000000\du)--(19.000000\du,15.000000\du)--(24.000000\du,15.000000\du)--(24.000000\du,13.000000\du)--cycle;
\pgfsetlinewidth{0.100000\du}
\pgfsetdash{}{0pt}
\pgfsetdash{}{0pt}
\pgfsetmiterjoin
\definecolor{dialinecolor}{rgb}{0.000000, 0.000000, 0.000000}
\pgfsetstrokecolor{dialinecolor}
\draw (19.000000\du,13.000000\du)--(19.000000\du,15.000000\du)--(24.000000\du,15.000000\du)--(24.000000\du,13.000000\du)--cycle;
% setfont left to latex
\definecolor{dialinecolor}{rgb}{0.000000, 0.000000, 0.000000}
\pgfsetstrokecolor{dialinecolor}
\node at (21.500000\du,14.195000\du){CS4516-03};
\definecolor{dialinecolor}{rgb}{1.000000, 1.000000, 1.000000}
\pgfsetfillcolor{dialinecolor}
\fill (13.000000\du,13.000000\du)--(13.000000\du,15.000000\du)--(18.000000\du,15.000000\du)--(18.000000\du,13.000000\du)--cycle;
\pgfsetlinewidth{0.100000\du}
\pgfsetdash{}{0pt}
\pgfsetdash{}{0pt}
\pgfsetmiterjoin
\definecolor{dialinecolor}{rgb}{0.000000, 0.000000, 0.000000}
\pgfsetstrokecolor{dialinecolor}
\draw (13.000000\du,13.000000\du)--(13.000000\du,15.000000\du)--(18.000000\du,15.000000\du)--(18.000000\du,13.000000\du)--cycle;
% setfont left to latex
\definecolor{dialinecolor}{rgb}{0.000000, 0.000000, 0.000000}
\pgfsetstrokecolor{dialinecolor}
\node at (15.500000\du,14.195000\du){CS4516-02};
\pgfsetlinewidth{0.100000\du}
\pgfsetdash{}{0pt}
\pgfsetdash{}{0pt}
\pgfsetmiterjoin
\pgfsetbuttcap
{
\definecolor{dialinecolor}{rgb}{0.000000, 0.000000, 0.000000}
\pgfsetfillcolor{dialinecolor}
% was here!!!
\pgfsetarrowsstart{stealth}
\pgfsetarrowsend{stealth}
{\pgfsetcornersarced{\pgfpoint{0.000000\du}{0.000000\du}}\definecolor{dialinecolor}{rgb}{0.000000, 0.000000, 0.000000}
\pgfsetstrokecolor{dialinecolor}
\draw (19.489738\du,8.885820\du)--(19.489738\du,10.942910\du)--(21.500000\du,10.942910\du)--(21.500000\du,13.000000\du);
}}
\pgfsetlinewidth{0.100000\du}
\pgfsetdash{}{0pt}
\pgfsetdash{}{0pt}
\pgfsetmiterjoin
\pgfsetbuttcap
{
\definecolor{dialinecolor}{rgb}{0.000000, 0.000000, 0.000000}
\pgfsetfillcolor{dialinecolor}
% was here!!!
\pgfsetarrowsstart{stealth}
\pgfsetarrowsend{stealth}
{\pgfsetcornersarced{\pgfpoint{0.000000\du}{0.000000\du}}\definecolor{dialinecolor}{rgb}{0.000000, 0.000000, 0.000000}
\pgfsetstrokecolor{dialinecolor}
\draw (17.558038\du,8.885820\du)--(17.558038\du,10.942910\du)--(15.500000\du,10.942910\du)--(15.500000\du,13.000000\du);
}}
\pgfsetlinewidth{0.100000\du}
\pgfsetdash{}{0pt}
\pgfsetdash{}{0pt}
\pgfsetmiterjoin
\pgfsetbuttcap
{
\definecolor{dialinecolor}{rgb}{0.000000, 0.000000, 0.000000}
\pgfsetfillcolor{dialinecolor}
% was here!!!
\pgfsetarrowsstart{stealth}
\pgfsetarrowsend{stealth}
{\pgfsetcornersarced{\pgfpoint{0.000000\du}{0.000000\du}}\definecolor{dialinecolor}{rgb}{0.000000, 0.000000, 0.000000}
\pgfsetstrokecolor{dialinecolor}
\draw (16.739230\du,8.560660\du)--(15.369615\du,8.560660\du)--(15.369615\du,10.000000\du)--(14.000000\du,10.000000\du);
}}
% setfont left to latex
\definecolor{dialinecolor}{rgb}{0.000000, 0.000000, 0.000000}
\pgfsetstrokecolor{dialinecolor}
\node[anchor=west] at (10.218800\du,7.369360\du){};
\end{tikzpicture}

            \label{fig:startop}
            \caption{``Star'' Network Topology}
        \end{subfigure}
        \hfill
        \begin{subfigure}[b]{0.4\textwidth}
            \centering
            % Graphic for TeX using PGF
% Title: /home/jason/Dropbox/CS/CS4516/Term_Project/images/topology1.dia
% Creator: Dia v0.97.2
% CreationDate: Sun Mar  2 22:43:32 2014
% For: jason
% \usepackage{tikz}
% The following commands are not supported in PSTricks at present
% We define them conditionally, so when they are implemented,
% this pgf file will use them.
\ifx\du\undefined
  \newlength{\du}
\fi
\setlength{\du}{15\unitlength}
\begin{tikzpicture}
\pgftransformxscale{1.000000}
\pgftransformyscale{-1.000000}
\definecolor{dialinecolor}{rgb}{0.000000, 0.000000, 0.000000}
\pgfsetstrokecolor{dialinecolor}
\definecolor{dialinecolor}{rgb}{1.000000, 1.000000, 1.000000}
\pgfsetfillcolor{dialinecolor}
\definecolor{dialinecolor}{rgb}{1.000000, 1.000000, 1.000000}
\pgfsetfillcolor{dialinecolor}
\fill (23.000000\du,9.000000\du)--(23.000000\du,11.000000\du)--(28.000000\du,11.000000\du)--(28.000000\du,9.000000\du)--cycle;
\pgfsetlinewidth{0.100000\du}
\pgfsetdash{}{0pt}
\pgfsetdash{}{0pt}
\pgfsetmiterjoin
\definecolor{dialinecolor}{rgb}{0.000000, 0.000000, 0.000000}
\pgfsetstrokecolor{dialinecolor}
\draw (23.000000\du,9.000000\du)--(23.000000\du,11.000000\du)--(28.000000\du,11.000000\du)--(28.000000\du,9.000000\du)--cycle;
% setfont left to latex
\definecolor{dialinecolor}{rgb}{0.000000, 0.000000, 0.000000}
\pgfsetstrokecolor{dialinecolor}
\node at (25.500000\du,10.195000\du){CS4516-04};
\definecolor{dialinecolor}{rgb}{1.000000, 1.000000, 1.000000}
\pgfsetfillcolor{dialinecolor}
\pgfpathellipse{\pgfpoint{18.523888\du}{12.500000\du}}{\pgfpoint{2.523888\du}{0\du}}{\pgfpoint{0\du}{1.500000\du}}
\pgfusepath{fill}
\pgfsetlinewidth{0.100000\du}
\pgfsetdash{}{0pt}
\pgfsetdash{}{0pt}
\pgfsetmiterjoin
\definecolor{dialinecolor}{rgb}{0.000000, 0.000000, 0.000000}
\pgfsetstrokecolor{dialinecolor}
\pgfpathellipse{\pgfpoint{18.523888\du}{12.500000\du}}{\pgfpoint{2.523888\du}{0\du}}{\pgfpoint{0\du}{1.500000\du}}
\pgfusepath{stroke}
% setfont left to latex
\definecolor{dialinecolor}{rgb}{0.000000, 0.000000, 0.000000}
\pgfsetstrokecolor{dialinecolor}
\node at (18.523888\du,12.695000\du){CS4516-05};
\definecolor{dialinecolor}{rgb}{1.000000, 1.000000, 1.000000}
\pgfsetfillcolor{dialinecolor}
\pgfpathellipse{\pgfpoint{18.523888\du}{7.500000\du}}{\pgfpoint{2.523888\du}{0\du}}{\pgfpoint{0\du}{1.500000\du}}
\pgfusepath{fill}
\pgfsetlinewidth{0.100000\du}
\pgfsetdash{}{0pt}
\pgfsetdash{}{0pt}
\pgfsetmiterjoin
\definecolor{dialinecolor}{rgb}{0.000000, 0.000000, 0.000000}
\pgfsetstrokecolor{dialinecolor}
\pgfpathellipse{\pgfpoint{18.523888\du}{7.500000\du}}{\pgfpoint{2.523888\du}{0\du}}{\pgfpoint{0\du}{1.500000\du}}
\pgfusepath{stroke}
% setfont left to latex
\definecolor{dialinecolor}{rgb}{0.000000, 0.000000, 0.000000}
\pgfsetstrokecolor{dialinecolor}
\node at (18.523888\du,7.695000\du){CS4516-01};
\definecolor{dialinecolor}{rgb}{1.000000, 1.000000, 1.000000}
\pgfsetfillcolor{dialinecolor}
\fill (13.000000\du,17.000000\du)--(13.000000\du,19.000000\du)--(18.000000\du,19.000000\du)--(18.000000\du,17.000000\du)--cycle;
\pgfsetlinewidth{0.100000\du}
\pgfsetdash{}{0pt}
\pgfsetdash{}{0pt}
\pgfsetmiterjoin
\definecolor{dialinecolor}{rgb}{0.000000, 0.000000, 0.000000}
\pgfsetstrokecolor{dialinecolor}
\draw (13.000000\du,17.000000\du)--(13.000000\du,19.000000\du)--(18.000000\du,19.000000\du)--(18.000000\du,17.000000\du)--cycle;
% setfont left to latex
\definecolor{dialinecolor}{rgb}{0.000000, 0.000000, 0.000000}
\pgfsetstrokecolor{dialinecolor}
\node at (15.500000\du,18.195000\du){CS4516-03};
\definecolor{dialinecolor}{rgb}{1.000000, 1.000000, 1.000000}
\pgfsetfillcolor{dialinecolor}
\fill (19.000000\du,17.000000\du)--(19.000000\du,19.000000\du)--(24.000000\du,19.000000\du)--(24.000000\du,17.000000\du)--cycle;
\pgfsetlinewidth{0.100000\du}
\pgfsetdash{}{0pt}
\pgfsetdash{}{0pt}
\pgfsetmiterjoin
\definecolor{dialinecolor}{rgb}{0.000000, 0.000000, 0.000000}
\pgfsetstrokecolor{dialinecolor}
\draw (19.000000\du,17.000000\du)--(19.000000\du,19.000000\du)--(24.000000\du,19.000000\du)--(24.000000\du,17.000000\du)--cycle;
% setfont left to latex
\definecolor{dialinecolor}{rgb}{0.000000, 0.000000, 0.000000}
\pgfsetstrokecolor{dialinecolor}
\node at (21.500000\du,18.195000\du){CS4516-02};
\pgfsetlinewidth{0.100000\du}
\pgfsetdash{}{0pt}
\pgfsetdash{}{0pt}
\pgfsetmiterjoin
\pgfsetbuttcap
{
\definecolor{dialinecolor}{rgb}{0.000000, 0.000000, 0.000000}
\pgfsetfillcolor{dialinecolor}
% was here!!!
\pgfsetarrowsstart{stealth}
\pgfsetarrowsend{stealth}
{\pgfsetcornersarced{\pgfpoint{0.000000\du}{0.000000\du}}\definecolor{dialinecolor}{rgb}{0.000000, 0.000000, 0.000000}
\pgfsetstrokecolor{dialinecolor}
\draw (16.739230\du,13.560660\du)--(16.739230\du,15.280330\du)--(15.500000\du,15.280330\du)--(15.500000\du,17.000000\du);
}}
\pgfsetlinewidth{0.100000\du}
\pgfsetdash{}{0pt}
\pgfsetdash{}{0pt}
\pgfsetmiterjoin
\pgfsetbuttcap
{
\definecolor{dialinecolor}{rgb}{0.000000, 0.000000, 0.000000}
\pgfsetfillcolor{dialinecolor}
% was here!!!
\pgfsetarrowsstart{stealth}
\pgfsetarrowsend{stealth}
{\pgfsetcornersarced{\pgfpoint{0.000000\du}{0.000000\du}}\definecolor{dialinecolor}{rgb}{0.000000, 0.000000, 0.000000}
\pgfsetstrokecolor{dialinecolor}
\draw (20.308500\du,13.560700\du)--(20.308500\du,15.280300\du)--(21.500000\du,15.280300\du)--(21.500000\du,17.000000\du);
}}
\pgfsetlinewidth{0.100000\du}
\pgfsetdash{}{0pt}
\pgfsetdash{}{0pt}
\pgfsetbuttcap
{
\definecolor{dialinecolor}{rgb}{0.000000, 0.000000, 0.000000}
\pgfsetfillcolor{dialinecolor}
% was here!!!
\pgfsetarrowsstart{stealth}
\pgfsetarrowsend{stealth}
\definecolor{dialinecolor}{rgb}{0.000000, 0.000000, 0.000000}
\pgfsetstrokecolor{dialinecolor}
\draw (17.558000\du,8.885820\du)--(17.558000\du,11.114200\du);
}
\pgfsetlinewidth{0.100000\du}
\pgfsetdash{}{0pt}
\pgfsetdash{}{0pt}
\pgfsetbuttcap
{
\definecolor{dialinecolor}{rgb}{0.000000, 0.000000, 0.000000}
\pgfsetfillcolor{dialinecolor}
% was here!!!
\pgfsetarrowsstart{stealth}
\pgfsetarrowsend{stealth}
\definecolor{dialinecolor}{rgb}{0.000000, 0.000000, 0.000000}
\pgfsetstrokecolor{dialinecolor}
\draw (19.489700\du,8.885820\du)--(19.489700\du,11.114200\du);
}
\pgfsetlinewidth{0.100000\du}
\pgfsetdash{}{0pt}
\pgfsetdash{}{0pt}
\pgfsetmiterjoin
\pgfsetbuttcap
{
\definecolor{dialinecolor}{rgb}{0.000000, 0.000000, 0.000000}
\pgfsetfillcolor{dialinecolor}
% was here!!!
\pgfsetarrowsstart{stealth}
\pgfsetarrowsend{stealth}
{\pgfsetcornersarced{\pgfpoint{0.000000\du}{0.000000\du}}\definecolor{dialinecolor}{rgb}{0.000000, 0.000000, 0.000000}
\pgfsetstrokecolor{dialinecolor}
\draw (20.855656\du,8.074025\du)--(21.927828\du,8.074025\du)--(21.927828\du,10.000000\du)--(23.000000\du,10.000000\du);
}}
% setfont left to latex
\definecolor{dialinecolor}{rgb}{0.000000, 0.000000, 0.000000}
\pgfsetstrokecolor{dialinecolor}
\node[anchor=west] at (18.523900\du,12.500000\du){};
% setfont left to latex
\definecolor{dialinecolor}{rgb}{0.000000, 0.000000, 0.000000}
\pgfsetstrokecolor{dialinecolor}
\node[anchor=west] at (10.218800\du,7.369360\du){};
\end{tikzpicture}

            \label{fig:looptop}
            \caption{``Loop'' Network Topology}
        \end{subfigure}
        \label{fig:topologies}
        \caption{Network Testing Topologies}
    \end{figure}
    Our final test will simulates a link failure in one of the redundant links connecting a LoopAware device and a conventional switch.

    % %Broadcast traffic moving average to show broadcast storm
    To demonstrate the efficacy of LoopAware in preventing broadcast storms, we measured the percent of total traffic comprised of broadcast frames when a forwarding loop existed in the network.
    The measurements were made by recalculating an exponentially weighted moving average whenever a new frame was sent by a switch.
    The average was then recorded in a file with the time (since the first packet was forwarded).
    Data and results of this experiment can be found in Section \ref{sec:results}.

    % Throughput measure using HTTP (with loop and without loop)
    We tested the throughput over our simulated links to show the performance of LoopAware.
    To test the throughput, we set up a simple python web server to serve up a 14 Megabyte file (consisting of random text) and downloaded it across two switches using curl.
    When the file finished downloading we recorded the download time and average download speed recorded by curl.
    Data and results from this test can be found in Section \ref{sec:results}.

    We also tested the response of LoopAware to failures of its redundant links.
    This test was conducted using a LoopAware device and a conventional Ethernet switch linked on two interfaces.
    Hosts on both sides of the redundant links sent ICMP ping requests to each other to show requests successfully crossing the
    interface.
    One connection was then severed (by deactivating an interface on one of the switches), and the response of the LoopAware device was observed.
    The results of this test are discussed in Section \ref{sec:results}.

    These tests demonstrate the functionality of the LoopAware and its backwards compatibility with non-LoopAware switches.
    Because simulation is done in software, many of our performance results may not accurately reflect the behavior of a hardware switch.
    We did not conduct simulations of hardware due to time constraints on our research.

\section{Results}
\label{sec:results}
  We collected performance metrics on networks utilizing our switching approach to measure its effectiveness.
  We also measured the performance of an unmodified software switch to allow direct comparison between approaches without encountering issues with    sources of error that may have been introduced by our simulation.
  Due to the fact that neither switch will be implemented in hardware, our performance results may be slightly different because hardware allows for greater parallelism.
  We measured the performance of the network when a link is at saturation to determine the benefit that a redundant bandwidth-sharing link is able to provide.
  We also measured and compared at the goodput of the network in these situations with and without congestion control to show the actual practical effect of our device on end hosts in the network.

  \subsection{Broadcast Storm Mitigation}
    \begin{figure}[ht]
	    \centering
	    \begin{subfigure}[b]{0.4\textwidth}
		    \centering
		    \begin{tikzpicture}
		    \begin{axis} [
			    title=Conventional Switch,
			    xlabel=Time ($\mu$s),
			    ylabel=\% Broadcast Traffic,
		    ]
		    \addplot+[mark size=0.6pt] table {../data/broadcast_storm_normal.dat};
		    \end{axis}
		    \end{tikzpicture}
		    \label{fig:stdbcast}
		    \caption{}
	    \end{subfigure}
	    \hfill
	    \begin{subfigure}[b]{0.4\textwidth}
		    \centering
		    \begin{tikzpicture}
		    \begin{axis} [
			    title=Smart Switch,
			    xlabel=Time ($\mu$s),
			    ylabel=\% Broadcast Traffic,
		    ]
		    \addplot+[mark size=0.6pt] table {../data/broadcast_storm_smart.dat};
		    \end{axis}
		    \end{tikzpicture}
		    \label{fig:smtbcast}
		    \caption{}
	    \end{subfigure}
	    \label{fig:bcast}
	    \caption{Percentage of Broadcast Traffic}
    \end{figure}

    To show the effect of broadcast storms on network traffic, we determined the percentage of broadcast frames from total network traffic both with and without a forwarding loop.
    Normally (in a star network with conventional switches), nearly all of the traffic will start as broadcast traffic during the auto-configuration period.
    The amount of broadcast traffic will then drop down to a noise-floor level.
    If the network has a loop in it, the amount of broadcast traffic will not drop off significantly because the network will start a broadcast storm as a result of the forwarding loop.
    Our device exhibits the same drop-off behavior even in a graph network because it filters broadcast packets out of forwarding loops, preventing a broadcast storm from occurring.

    \begin{table}[ht]
	    \centering
	    \caption{File Transfer Statistics}
	    \label{tab:throughput}
	    \begin{tabular}{|c|c|c|c|}
		    \hline
		    \multicolumn{2}{|c|}{} & With Loop	& Without Loop \\
		    \hline
		    \multirow{2}{*}{Conventional Switch}& Throughput	& 181 kBps	& 201 kBps \\ \cline{2-4}
		    & Time	& 0:01:16	& 0:01:08 \\
		    \hline
		    \multirow{2}{*}{Smart Switch}	& Throughput	& 223 kBps	& 191 kBps \\ \cline{2-4}
		    & Time	& 0:01:01	& 0:01:11\\
		    \hline
	    \end{tabular}
    \end{table}

    Preventing broadcast storms has a noticeable effect on the performance of the network from the perspective of end hosts.
    The average goodput of the network when a forwarding loop exists in the network is significantly improved by our device because it prevents the broadcast storm from occurring.
    These findings are not terribly surprising, but do confirm that the device is capable of preventing broadcast storms without having to disable links.
    This ability is key to the more novel features of the device, bandwidth sharing and seamless fail-over.

  \subsection{Congestion Conditions}
    Situations links in the network are at saturation show the most significant difference in performance between architectures.
    Setups with redundant links perform much better than any other system when there is one flow causing all or most of the congestion, because all of  the other traffic defaults to the other link.
    Single links are unable to provide this service, and as a result they provide markedly worse performance in congestion conditions.

  \subsection{Congestion Control}
    We do not yet know what will happen in this situation.
    The results will come from having hosts send traffic to each other with the default TCP congestion control mechanism in Linux 3.12.9.
    Testing with congestion control may expose flaws in our bandwidth sharing technique, because links only hover near saturation, so the switch may have a hard time determining that more flows need to be moved over to the other link, because the flows will be too interdependent.

  \subsection{Response Time}
    Although our device does require additional processing of broadcast frames, non-broadcast frames can be handled without much additional overhead because they can be quickly found in the flow table.
    Results of our testing indicate that the average response time is not significantly affected by the addition of intelligent switching hardware.
    The advantages of using our device are not offset by any increase in response time from processing, except initially when the flow table is being constructed and when legitimate broadcast traffic is much higher than normal.

\documentclass{article}
\usepackage[T1]{fontenc}
\usepackage[utf8]{inputenc}
\usepackage[section]{placeins}
\usepackage{fullpage}
\usepackage{graphicx}
\usepackage{caption}
\usepackage{subcaption}
\usepackage{hyperref}

\hypersetup{colorlinks=true,
			urlcolor=blue,
			linktoc=all,
			linkcolor=black,
			citecolor=black}

\graphicspath{ {./images/} }

\title{CS 4516 Group \#5: Bandwidth Trunking Using Layer 2 Devices\\Conclusion}
\author{Jason Rosenman \and Louis Fogel \and Sam Abradi}
\date{}

\begin{document}
\maketitle


\end{document}

\newpage
\bibliographystyle{plain}
\bibliography{Bibliography}
\end{document}
