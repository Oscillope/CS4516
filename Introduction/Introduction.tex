\section{Introduction}
  \subsection{Need}
	  Except for a in few special applications, Ethernet is far and away the most popular choice of link layer protocol.
	  In consumer and enterprise level networks there are no real viable alternatives.
	  Ethernet's popularity stems from its simplicity and versatility, but it is not without shortcomings.

	  Typical modern installations of switched Ethernet are highly susceptible to broadcast storms created by forwarding loops.
	  This problem is especially important in networks with publicly accessible infrastructure such as schools and hotels.
	  Intelligent switching hardware could prevent these broadcast storms from occurring, and additionally, could provide better reliability and reduce bottlenecks by allowing redundant links and bandwidth sharing between routes.
	
	  In commercial systems, reliability and fool-proofing are important.
	  With Ethernet as it currently stands (without intelligent switching hardware), a malicious or simply confused user could easily create broadcast storms or forwarding loops simply by connecting a switch to itself or directly to another switch that is already connected to that switch.
	  These storms could easily bring down all or part of a large network.
	  Therefore, there is a need for a system to address and resolve this security and performance risk.
	
	  Solutions to these problems are either too slow to allow full featured link redundancy or prohibitively expensive and involve entirely replacing network infrastructure.
	  Additionally, some of these devices depend on external configuration by administrators, incurring additional setup  and maintenance costs.
	  Our approach allows link sharing and broadcast storm mitigation in switched Ethernet networks using a single self-configuring device.
  \subsection{Approach}
	  Our team proposes the creation an Ethernet switch that implements flow-based forwarding.
	  Unlike a conventional switch, which uses a forwarding table to determine which interface to forward frames to, our system would create flows based on source IP address (when available) to allow bandwidth sharing over linked interfaces.
	  Even when linked with a conventional switch, our device could utilize multiple links to share bandwidth upstream (towards the linked switch) and have redundancy downstream (from the linked switch).
	  Assuming that the conventional switch would forward to multiple ports simultaneously if it detects the same MAC address multiple times, a link could fail between our switch and the conventional switch the loss of a single frame.
	  As a result, ongoing connections can be maintained throughout a link failure.

	  We propose detecting forwarding loops by caching a hash of each frame that is received, and comparing it to incoming frames to detect duplicates.
	  To increase efficiency and limit hardware complexity, we can use the CRC that is already included in the Ethernet frame as the hash.
	  We propose calculating the necessary size of the cache using estimated time-to-live (TTL) values for the CRCs.
	  The TTL would be a function of the maximum number of hops in a loop, and of the maximum amount of broadcast traffic.
	  These maximums would be practical limitations on the system we are interested in supporting, as opposed to theoretical limitations.
	  In the case that these maximums are exceeded, our device could fall back to simply disabling redundant links when a loop is detected, thereby turning the network into a directed acyclic graph.
	  The looped interface will remain disabled until the state of the network falls within the limits of the cache.
	
	  When no redundant links exist, the created flows would essentially form a forwarding table.
	  However, when redundant links are formed, the device could use the source IP address of the packet to make flows that determine forwarding strategy for frames.
	  We propose creating flows based on measured bandwidth of interfaces with the goal of uniform utilization.
	  Our device could detect link failures when a previously existing loop no longer exists.
	  In this condition, the device would delete all flows related to that loop.
	  This would cause it to rebuild new flows using the remaining available links.

	  To determine the feasibility of this approach to intelligent switching the team proposes simulation of the behavior of a network of intelligent switches in software.
	  The team would construct a simulated network consisting of virtual machines capable of creating network traffic.
	  The virtual machines will perform a series of predetermined requests through a virtual router to servers both inside and outside of the network.
	  We will also have servers on the virtual network that will receive requests from other hosts on the network as well as foreign hosts outside the network.
	  Several topologies and scenarios would be employed in testing our system to show a range of behaviors in the device.
	  We can also test with two conventional ``dumb'' switches in the same configuration, to demonstrate that a broadcast storm would occur under these conditions.
	  We propose verification of the functionality of our system by confirming that the requests and responses are correct.
	  The effectiveness of the system would be measured by gauging the latency and throughput of the network in forwarding the requests.
  \subsection{Benefit}
    First and foremost, intelligent switches will detect and prevent broadcast storms in any forwarding loop they are a part of.
	  Additionally, intelligent switches will be able to detect and avoid link failures and share bandwidth between redundant links to mitigate bottlenecks using the additional information they collect.
	  Because switches implementing the behavior we describe will still be entirely compliant with existing Ethernet networks, this approach will also allow switches that are not able to detect loops to operate in the same network.
	  If non-intelligent switches are used in combination with at least one intelligent switch and the network is small enough to allow CRC caching, our hardware should still be able to share bandwidth by essentially tricking the non-intelligent switch into thinking that it is two switches attached to separate ports.
	  Since our switches are auto-configuring as well as backwards compatible with existing hardware, implementing them in a network requires a very low installation and maintenance overhead.
	  Additionally, because no configuration is necessary and backward compatibility is ensured, our device can be installed on an operational network with little (if any) required downtime.
	
	  We believe our device would have appeal to administrators of small to medium sized enterprise networks.
	  When limited IT staff are available, the auto-configuration and robustness of our device are desirable.
	  Additionally, the backwards compatibility should keep cost of integration with existing networks low.
	  We believe there is a market for robust networks without the headaches of more complex solutions that are better suited for large networks and datacenters.
  \subsection{Competition}
	  Previous research has identified methods of detecting and eliminating routing loops in switched Ethernet networks but falls short of automatic network topology detection for link redundancy or bandwidth sharing.
	  Many proposed solutions require replacement of all network infrastructure, which is prohibitively expensive.
	  Additionally, some require manual configuration of the network, costing time and allowing for misconfiguration that could cause network outages.
	
	  One proposed solution to the problem of forwarding loops in layer 2 networks is Spanning Tree Protocol (STP).
	  STP works by sending special frames, known as Bridge Protocol Data Units (BPDUs), to identify the topology of the network and identify a spanning tree and disables all connections except those in the spanning tree.
	  STP has several shortcomings that make it an undesirable alternative to our solution.
	  Mainly, it requires replacing the entire network infrastructure with STP-capable hardware, which is expensive and necessitates significant network downtime.
	  In addition, STP works by disabling redundant connections, and as a result can not support bandwidth sharing.
	  Because our system does not disable links (except when overloaded), it is able to send traffic to the same host over different connections, preventing bottlenecks from forming in the forwarding graph.
    In some network topologies, the use of STP can result in count-to-infinity conditions between hosts, causing some frames to become unforwardable.
    Our device is not dependent on a hop count and is therefore immune to the count-to-infinity problem.
	  Because spanning tree detection in STP is dependent on the ability of switches to detect BPDUs, it does not completely prevent the possibility of forwarding loops.
	
	  Our device builds in part on the design of a similar system which detects forwarding loops in an Ethernet network using STP.
	  The inventors of this system named it "EtherFuse" as an analogy to current limiting electrical fuses \cite{etherfuse}.
	  EtherFuse uses its capability to detect forwarding loops to disable links that create forwarding loops, as opposed to increasing bandwidth between switches.
	  Our device expands on the forwarding loop detection technique that EtherFuse proved effective, but provides additional functionality that is useful for administrators of enterprise networks.
	  EtherFuse is only compatible with networks that already support Spanning Tree Protocol (STP).
	  This device, in addition to detecting loops, also keeps track of count-to-infinity conditions that can occur under STP.
	  In the condition of a loop or count-to-infinity, the device would simply sever the connection on which it is placed.
	  This has several disadvantages.
	  First, an EtherFuse device must be purchased for every link that may take part in a loop.
	  Second, the EtherFuse will only sever the connection, as opposed to building on it as we propose.
	
	  Another potential solution to the problems our proposed device is intended to solve is the replacement of network hardware with OpenFlow devices.
	  OpenFlow capable switching hardware allows for custom rules for the handling of flows instead of the traditional implementation of Ethernet packet forwarding \cite{openflow}.
	  While OpenFlow allows for an impressive number of features and behaviors, and indeed, could even be used as a testbed for our behaviors, it is better suited for research networks and data centers than enterprise networks.
	  The enhanced capability of OpenFlow comes at the high cost of replacing all network hardware with OpenFlow-capable devices.
	  The extended network downtime necessitated by such an extreme infrastructure upgrade would also probably prove unacceptable for enterprise networks.
	  Additionally, providing an OpenFlow device with the same rules that our proposed device would operate on requires manual configuration of the device (something our device would not require).
	  Finally, because flow control rules for OpenFlow are handled in software, an OpenFlow device that automatically prevents new loops from causing broadcast storms would be much slower to handle broadcast frames.

	  Our device aims to be ideal for enterprise networks by combining the best of both worlds; the robust auto-configuration and cost-effectiveness of EtherFuse with the powerful traffic management of OpenFlow.
	  Enterprise networks need to be cheap, reliable, and easy to maintain.
	  The device we propose solves the problems and limitations of switched Ethernet without compromising the cost-effectiveness of commodity hardware, the reliability of Ethernet, or the auto-configuration of traditional frame forwarding.
	  Our device allows for an easy migration to a more advanced layer 2 infrastructure without requiring a complete system overhaul.
