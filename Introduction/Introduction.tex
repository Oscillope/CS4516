\section{Introduction}
\label{sec:introduction}
	%Need
	This paper introduces LoopAware, a device that can be used to allow more complex topologies in Ethernet networks.
	Namely, it allows Ethernet networks to contain redundant links without creating broadcast storms or forwarding loops.
	Unlike other systems that can allow loops in Ethernet networks, LoopAware can use loops in networks for redundancy as well as bandwidth sharing.
	It is also auto-configuring, so it requires no user intervention in order to work.
	LoopAware is backwards compatible with non-LoopAware switches.
	
	Except for a in few special applications, Ethernet is far and away the most popular choice of link layer protocol.
	In consumer and enterprise level networks there are no real viable alternatives.
	Ethernet's popularity stems from its simplicity and versatility, but it is not without shortcomings.

	Typical modern installations of switched Ethernet are highly susceptible to broadcast storms created by forwarding loops.
	This problem is especially important in networks with publicly accessible infrastructure such as schools and hotels.
	Intelligent switching hardware could prevent these broadcast storms from occurring, and additionally, could provide better reliability and reduce bottlenecks by allowing redundant links and bandwidth sharing between routes.

	In commercial systems, reliability and fool-proofing are important.
	With Ethernet as it currently stands (without intelligent switching hardware), a malicious or simply confused user could easily create broadcast storms or forwarding loops simply by connecting a switch to itself or directly to another switch that is already connected to that switch.
	These storms could easily bring down all or part of a large network.
	Therefore, there is a need for a system to address and resolve this security and performance risk.
	
	%Competition
	Previous research has identified methods of detecting and eliminating routing loops in switched Ethernet networks but fails to automatically detect network topology for bandwidth sharing.
	Many proposed solutions require replacement of all network infrastructure, which is prohibitively expensive.
	Additionally, some require manual configuration of the network, costing time and allowing for misconfiguration that could cause network outages.

	One proposed solution to the problem of forwarding loops in layer 2 networks is Spanning Tree Protocol (STP).
	STP works by sending special frames, known as Bridge Protocol Data Units (BPDUs), to identify the topology of the network and identify a spanning tree and disables all connections except those in the spanning tree.
	STP has several shortcomings that make it an undesirable alternative to our solution.
	Mainly, it requires replacing the entire network infrastructure with STP-capable hardware, which is expensive and necessitates significant network downtime.
	In addition, STP works by disabling redundant connections, and as a result can not support bandwidth sharing.
	Because our system does not disable links (except when overloaded), it is able to send traffic to the same host over different connections, preventing bottlenecks from forming in the forwarding graph.
	In some network topologies, the use of STP can result in count-to-infinity conditions between hosts, causing some frames to become unforwardable.
	Our device is not dependent on a hop count and is therefore immune to the count-to-infinity problem.
	Because spanning tree detection in STP is dependent on the ability of switches to detect BPDUs, it does not completely prevent the possibility of forwarding loops.

	Our device builds in part on the design of a similar system, EtherFuse, which detects forwarding loops in an Ethernet network using STP.
	EtherFuse uses its capability to detect forwarding loops to disable redundant links in cases where STP fails to \cite{etherfuse}.
	Our device expands on the forwarding loop detection technique that EtherFuse proved effective, but provides additional functionality that is useful for administrators of enterprise networks.
	EtherFuse is only compatible with networks that already support Spanning Tree Protocol (STP).
	In addition to detecting loops, EtherFuse also keeps track of count-to-infinity conditions that can occur under STP.
	In the condition of a loop or count-to-infinity, the device would simply sever the connection on which it is placed.
	LoopAware makes several improvements to the functionality of EtherFuse.
	First, an EtherFuse device must be purchased for every link that may take part in a loop, while LoopAware allows the prevention of forwarding loops without the need for additional devices in certain topologies.
	Second, the EtherFuse will only sever the connection, as opposed to building on it as LoopAware does.

	OpenFlow is another technology that can be used to prevent forwarding loops in Ethernet networks.
	OpenFlow capable switching hardware allows for custom rules for the handling of flows instead of the traditional implementation of Ethernet packet forwarding \cite{openflow}.
	While OpenFlow allows for an impressive number of features and behaviors, and indeed, could even be used as a testbed for our behaviors, it is better suited to research networks and data centers than enterprise networks.
	The enhanced capability of OpenFlow comes at the high cost of replacing all network hardware with OpenFlow-capable devices.
	The extended network downtime necessitated by such an extreme infrastructure upgrade would also probably prove unacceptable for enterprise networks.
	Additionally, providing an OpenFlow device with the same rules that our proposed device would operate on requires manual configuration of the device (something our device would not require).
	Finally, because flow creation rules for OpenFlow are handled in software, an OpenFlow device that automatically prevents new loops from causing broadcast storms would be much slower to handle broadcast frames.

	LoopAware aims to be ideal for enterprise networks by combining the best of both worlds: the robust auto-configuration and cost-effectiveness of EtherFuse and the powerful traffic management of OpenFlow.
	Enterprise networks need to be cheap, reliable, and easy to maintain.
	LoopAware solves the problems and limitations of switched Ethernet without compromising the cost-effectiveness of commodity hardware, the reliability of Ethernet, or the auto-configuration of traditional frame forwarding.
	LoopAware allows for an easy migration to a more advanced layer 2 infrastructure without requiring a complete system overhaul.
	
	We believe LoopAware would have appeal to administrators of small to medium sized enterprise networks.
	When limited IT staff are available, the auto-configuration and robustness of our device are desirable.
	Additionally, the backwards compatibility should keep cost of integration with existing networks low.
	We believe there is a market for robust networks without the headaches of more complex solutions that are better suited for large networks and datacenters.
	
	In this paper we describe an implementation of a LoopAware device.
	The implementation is described in Section \ref{sec:approach}.
	We then test our implementation to demonstrate its efficacy and performance.
	The testing methodology is described in Section \ref{sec:methodology} and results are presented and analyzed in Section \ref{sec:results}.
	We conclude in Section \ref{sec:conclusion}.
